\documentclass[a4paper,12pt,twoside]{scrreprt}
% Autor der Vorlage: Klaus Rheinberger, FH Vorarlberg
% 2017-02-20

%% Hilfe: z.B.
% empfohlener Einstieg: http://latex.tugraz.at/
% https://de.wikibooks.org/wiki/LaTeX-Kompendium:_Schnellkurs:_Erste_Schritte
% https://de.wikibooks.org/wiki/LaTeX-Kompendium:_Schnellkurs
% https://de.wikibooks.org/wiki/LaTeX-Kompendium

%% Pakete:
% Der Befehl \usepackage[latin9]{inputenc} ermöglicht die direkte Angabe von Umlauten. Übrigens lässt sich so auch das Euro-Zeichen direkt eingeben. Auf Betriebssystemen, wie zum Beispiel allen neueren Linux-Distributionen, verwendet man statt \usepackage[latin9]{inputenc} besser \usepackage[utf8]{inputenc}, auf Applesystemen verwendet man \usepackage[macce]{inputenc} (oder das für ältere Modelle gültige \usepackage[applemac]{inputenc}).
\usepackage[utf8]{inputenc}
\usepackage[T1]{fontenc}    % Silbentrennung bei Sonderzeichen
\usepackage{graphicx}       % Bilder einbinden
\usepackage[english]{babel} % Deutsche Sprachanpassungen
\usepackage{csquotes}       % When using babel or polyglossia with biblatex, loading csquotes is recommended to ensure that quoted texts are typeset according to the rules of your main language.
\usepackage{acronym}  % für optionales Abkürzungsverzeichnis
\usepackage{eurosym}  % z. B. \EUR{12345,68}
\usepackage[linktocpage=true]{hyperref} % Links z. B. \href{https://www.wikibooks.org}{Wikibooks home}
\usepackage[bindingoffset=8mm]{geometry}% Bindeverlust von 8mm einbeziehen. Mit dem geometry-Paket können Sie die Ränder auch ganz individuell anpassen.
\usepackage{caption} % Abbildungslegenden
\captionsetup{format=hang, justification=raggedright}

\usepackage[a-2b,mathxmp]{pdfx}[2018/12/22]

\usepackage[style=ieee,citestyle=ieee,backend=biber]{biblatex}   % Literaturverweise
%\usepackage[style=numeric,citestyle=numeric,backend=biber]{biblatex}
% biblatex comes with a variety of built-in bibliography/citation style families (numeric, alphabetic, authoryear, authortitle, verbose), and there's a growing number of custom styles:
% https://de.sharelatex.com/learn/Biblatex_citation_styles
% https://de.sharelatex.com/learn/Biblatex_bibliography_styles
\addbibresource{Zotero-Beispiele.bib}    % Zotero-Beispiele.bib ist die verwendete Bibtex-Datei
% Anstatt die Bibtex-Datei selber zu erstellen, kann sie z. B. aus einer Zotero-Sammlung zu BibTeX exportiert werden.


%% Einstellungen:
\setcounter{secnumdepth}{4}
\setcounter{tocdepth}{4}   % Tiefe der Gliederung im In haltsverzeichnis


%% ERSETZEN VON ECKIGEN KLAMMERN:
% Ersetzen Sie den Text in den eckigen Klammern!

\begin{document}

% evtl. Sperrvermerkseite
\thispagestyle{empty}

\noindent
[Achtung: Verwenden Sie einen Sperrvermerk nur in sehr gut begründeten Fällen kekk mekk spek!]

\section*{[evtl. Sperrvermerk]}   % evtl. ersetzen durch \section*{Sperrvermerk}
Die vorliegende Arbeit ist bis zum [DATUM] für die öffentliche Nutzung zu sperren. Veröffentlichung, Vervielfältigung und Einsichtnahme sind ohne meine ausdrückliche Genehmigung nicht gestattet. Der Titel der Arbeit sowie das Kurzreferat/Abstract dürfen veröffentlicht werden.

\vspace{3cm}

\noindent Dornbirn, \hfill Unterschrift Verfasser*in


% Titelblatt:
% \newpage\mbox{}\newpage
\cleardoublepage   % force output to a right page
\thispagestyle{empty}
\begin{titlepage}
  \begin{flushright}
  \includegraphics[width=0.4\linewidth]{Abbildungen/Wort-Bild-Marke-cmyk}  % https://www.fhv.at/fh/presse/logo-bildmaterial
  \end{flushright}
  \begin{flushleft}
  \section*{Classification of GPS Track Data Using AI Methods}
  \subsection*{A Case Study of Waste Collection Vehicles}
  \vspace{1cm}

  Bachelor thesis\\
  for obtaining the academic degree
  \vspace{0.5cm}

  \textbf{Bachelor of Science in Engineering (BSc)}

  \vspace{1cm}
  Vorarlberg University of Applied Sciences\newline
  Computer Science - Software and Information Engineering

  \vspace{0.5cm}

  Supervised by\newline
  Dipl.-Ing. Dr. techn. Ralph Hoch

  \vspace{0.5cm}

  Submitted by\newline
  Matthias Hefel\newline
  Dornbirn, May 2025
  \end{flushleft}
\end{titlepage}


% evtl. Widmung:
\newpage

\section*{Dedication}

\vspace{1cm} 
\begin{center}
    \emph{Dedicated to my younger self, who never stopped chasing his dream and never will!}\\[0.5cm]
    \emph{And to my parents, who supported me throughout this journey.}\\[0.5cm]
    \textbf{Thank you.}
\end{center}
\vspace{1cm}

% Kurzreferat:
\newpage
\section*{Kurzreferat}

\subsection*{Klassifizierung von GPS-Spurdaten mit Unterstützung von KI-Methoden am Beispiel von Abfallsammelfahrzeugen}

In der Abfallwirtschaft ist die strategische Tourenplanung ein wichtiger Prozess, in dem durch optimale Gebietsaufteilung eine maximal effiziente Fuhrparkauslastung bei möglichst geringen Kosten ermittelt wird. Dies geschieht in Entsorgungsbetrieben sowohl für bestehende Auftragsgebiete, als auch bei der Kalkulation von neuen Ausschreibungen. Vor Allem bei Regionen, in denen keine Erfahrungswerte vorliegen müssen für eine robuste Tourenplanung zahlreiche unscharfe Annahmen getroffen und manchmal auch Schätzungen vorgenommen werden. Um diese Unsicherheiten durch die Analyse von geographischen Strukturen zur verringern soll eine Technologie in die bestehende Tourenplanungssoftware der Firma integriert werden, die folgende Aufgabenstellung automatisiert lösen kann: Anhand von bestehenden GPS-Aufzeichnungen sollen strukturelle Eigenschaften der jeweilige Sammelgebiete numerisch bewertet und klassifiziert werden. Gleichermaßen sollen anhand von geographischen (und möglichst frei verfügbaren Strukturdaten) aus noch unbekannten Gebieten erhoben werden können um diese auf die selbe Art und Weise klassifizieren zu können. Dadurch entsteht einerseits eine Referenzdatenmenge (von bestehenden Sammeltouren) und eine Vergleichsdatenmenge (aus den neuen Ausschreibungsgebieten). Dort wo die Klassifizierungsdaten übereinstimmen, kann davon ausgegangen werden, dass die planungsrelevanten Kennzahlen aus bestehenden Auftragsgebieten ohne gewagte Annahmen einfach übernommen werden können. Die Klassifizierung von GPS-Daten und geographischen Strukturdaten soll mit Hilfe von künstlicher Intelligenz automatisiert erstellt werden können. Auch die Überlegung, welche geographischen Strukturdaten denn überhaupt aussagekräftig sind um einen Vergleich anzustreben, sollen ggf. mit Hilfe von KI Technologien erfolgen. 

Das Ziel der praktischen Arbeit ist es einen Sandbox-Service zu implementieren, der von der bestehenden Software der infeo aufgerufen und mit Daten befüllt werden kann um so "auf Knopfdruck" Klassifizierungen und Vergleiche von GPS-Daten und Ausschreibungs-Strukturdaten zu erstellen. Die Anwender:innen haben dadurch die Möglichkeit für neue Ausschreibungen entsprechend passende Planungsparameter aus ihren bestehenden Auftragsgebieten zu berechnen und somit die Unsicherheiten bei der Ausschreibungskalkulation deutlich zu reduzieren.

\vspace{0.5cm}

\noindent
[Evtl. 5-7 Schlagwörter:]

% Abstract:
\newpage
\section*{Abstract}
\subsection*{Classification of GPS Track Data Using AI Methods: A Case Study of Waste Collection Vehicles}

In waste management, strategic route planning is a crucial process where optimal fleet utilization is determined through the efficient division of service areas, with the goal of minimizing costs. This process is applied by waste disposal companies both for existing service areas and when calculating bids for new tenders. Especially in regions where there is no prior experience, numerous uncertain assumptions and estimates must be made for robust route planning. To reduce these uncertainties through the analysis of geographical structures, a technology will be integrated into the company’s existing route planning software, which can automatically solve the following task: Based on existing GPS records, the structural characteristics of the respective collection areas should be numerically evaluated and classified. Additionally, geographical structural data (preferably from freely available sources) from unknown areas should be collected and classified in the same way. This approach will create both a reference data set (from existing collection routes) and a comparison data set (from new tender areas). Where the classification data match, it can be assumed that planning-relevant parameters from existing service areas can be applied to the new areas without risky assumptions. The classification of GPS data and geographical structural data should be automated using artificial intelligence. Furthermore, the consideration of which geographical structural data are meaningful for comparison should, if necessary, also be supported by AI technologies.

The practical goal of this work is to implement a sandbox service that can be called and populated with data by the existing software of infeo, enabling the creation of classifications and comparisons of GPS data and tender structural data "at the push of a button." This will provide users with the ability to calculate appropriate planning parameters from their existing service areas for new tenders, thereby significantly reducing uncertainties in bid calculations.
\vspace{0.5cm}

\noindent
[Optionally 5-7 keywords:]

% evtl. Vorwort:
\newpage
\section*{Preface}   % evtl. ersetzen durch \section*{Widmung}

[Text des Vorworts]


% Inhaltsverzeichnis:
\cleardoublepage   % force output to a right page
\tableofcontents

\clearpage
\phantomsection
\addcontentsline{toc}{chapter}{List of Figures}
\listoffigures

\clearpage
\phantomsection
\addcontentsline{toc}{chapter}{List of Tables}
\listoftables

% evtl. Abkürzungsverzeichnis:
\clearpage
\phantomsection
\addcontentsline{toc}{chapter}{List of Abbreviations}  % evtl. ersetzen durch \addcontentsline{toc}{chapter}{Abkürzungsverzeichnis}
\chapter*{List of Abbreviations} % evtl. ersetzen durch \chapter*{Abkürzungsverzeichnis}
\begin{acronym}[SQL]
 \acro{GPS}{Global Positioning System}
 \acro{SQL}{Structured Query Language}
 \acro{Bash}{Bourne-again shell}
\end{acronym}

%% Die Kapitelstruktur ist mit der betreuungsperson abzustimmen!

\chapter{Introduction}
Formatvorlage für den Fließtext. Formatvorlage für den Fließtext. Formatvorlage für den Fließtext. Formatvorlage für den Fließtext. Formatvorlage für den Fließtext. Formatvorlage für den Fließtext. Formatvorlage für den Fließtext. Formatvorlage für den Fließtext. Formatvorlage für den Fließtext. Formatvorlage für den Fließtext. Formatvorlage für den Fließtext.
\begin{quote}
  Formatvorlage für ein längeres direktes Zitat. Formatvorlage für ein längeres direktes Zitat. Formatvorlage für ein längeres direktes Zitat. Formatvorlage für ein längeres direktes Zitat. Formatvorlage für ein längeres direktes Zitat. Formatvorlage für ein längeres direktes Zitat….
\end{quote}

\EUR{12345,68}, \href{https://www.wikibooks.org}{Wikibooks home}

\section{Problem Statement}

\section{Motivation}

\section{Solution Approach}

\section{Structure of the Work}


\chapter{Background and Related Work}
Formatvorlage für den Fließtext.
Hier eine Liste.
\begin{enumerate}
 \item Verstehen
 \item Üben
 \item Können
\end{enumerate}


\section{Technical Background}
Formatvorlage für den Fließtext. Die Abbildung~\ref{fig:ex} auf Seite \pageref{fig:ex} zeigt drei Entladungskurven eines biphasischen Defibrillators.
\begin{figure}[htb]
  \centering
  \includegraphics[width=10cm]{Abbildungen/Amann_TechnAbb}
  \caption[Aufheizverhalten von PTFE]{Aufheizverhalten von PTFE. \\Quelle: eigene Ausarbeitung}
 \label{fig:ex}
\end{figure}


\section{Related Work}
Formatvorlage für den Fließtext.
Jetzt eine Fußnote\footnote{Dies ist eine Fußnote.}
Die quadratischen Gleichung (\ref{equ:foo}) hat wieviele Nullstellen?
\begin{equation}
 \label{equ:foo}
 x^2-2x+5=0.
\end{equation}
Zwei von Einsteins berühmtesten Formeln lauten:
\begin{eqnarray*}
  E &= mc^2                                  \\
  m &= \frac{m_0}{\sqrt{1-\frac{v^2}{c^2}}}
\end{eqnarray*}


\subsection{[Unterkapitel dritte Ebene]}
Formatvorlage für den Fließtext. Hier die einfache Tabelle \ref{tab:sp}

\begin{table}[htb]
  \centering
  \begin{tabular}{ | l | l |c|}
    \hline
    Datum      & Thema           & Raum \\
    \hline\hline
    Montag     & Graphentheorie  & U1   \\
    \hline
    Donnerstag & Algebra         & MZB23\\
    \hline
  \end{tabular}
  \caption[Stundenplan]{Stundenplan des Jahres 2030.\\Quelle: eigene Ausarbeitung}
  \label{tab:sp}
\end{table}

\subsubsection{[Unterkapitel vierte Ebene]}
Formatvorlage für den Fließtext.



\section{[Unterkapitel zweite Ebene]}

Verweise: zu einem Buch mit Details \cite[vgl.][Kapitel 2]{bathe_finite-elemente-methoden_1990} oder ohne Details \cite{bathe_finite-elemente-methoden_1990}, ein Buchteil \cite{areger_problem-based_2007}, eine Dissertation \cite{sporn_interaktives_2000}, ein Dokument \cite{industriellenvereinigung_beste_2014}, ein Enzyklopädieartikel \cite{brockhaus_kreativitat_1872}, ein Film \cite{de_wilde_through_2008}, ein Konferenz-Paper \cite{weber_podcasts._2006}, ein Magazin-Artikel \cite{autornachname1_magazinartikeltitel_1995}, ein Pordcast \cite{paulus_horen_????}, eine Tonaufnahme \cite{horowitz_horowitz_2003}, eine Videoaufnahme \cite{fhvlearningsupport_was_2008}, ein Vortrag \cite{kohls_literaturverwaltung_2008}, eine Website \cite{wedekind_von_2008}, ein Zeitschriftenartikel \cite{hofer_wir_2008} und ein Zeitungsartikel \cite{schenkel_tsunami_2012}.


\chapter{Problem Definition and Solution Approach}

\section{Description of Datasets}
Formatvorlage für den Fließtext.

\subsection{[Unterkapitel dritte Ebene]}
Formatvorlage für den Fließtext.

\subsubsection{[Unterkapitel vierte Ebene]}
Formatvorlage für den Fließtext.

\section{Big Picture}
Formatvorlage für den Fließtext.

\section{Solution Approach}
Formatvorlage für den Fließtext.


\chapter{Implementation}

\section{Implementation of the Big Picture}
Formatvorlage für den Fließtext.

\section{Integration with existing systems}
Formatvorlage für den Fließtext.


\chapter{Evaluation and Discussion}

\section{Definition of the data sets used for the evaluation}
Formatvorlage für den Fließtext.

\section{Evaluation of the results}
Formatvorlage für den Fließtext.

\section{Reflection on the results}
Formatvorlage für den Fließtext.


\chapter{Conclusion}

\section{Future Directions}
Formatvorlage für den Fließtext.

\section{Limitations}
Formatvorlage für den Fließtext.


% Literaturverzeichnis:
\clearpage
\phantomsection
\addcontentsline{toc}{chapter}{Literaturverzeichnis}
\printbibliography


\chapter*{[evtl. Anhang]}  % evtl. ersetzen mit \chapter*{Anhang}
\addcontentsline{toc}{chapter}{[evtl. Anhang]}   % evtl. ersetzen mit \addcontentsline{toc}{chapter}{Anhang}
Formatvorlage für den Fließtext.


\chapter*{Affidavit}
\addcontentsline{toc}{chapter}{Affidavit}
I hereby declare in lieu of oath that I have written this Bachelor
thesis independently and without the use of aids other than those specified.
aids other than those specified. The passages taken directly or indirectly from other sources
directly or indirectly from other sources are marked as such. The thesis has not been
neither in the same nor in a similar form to any other examination authority
nor has it been published.

\vspace{3cm}
\noindent
Dornbirn, on 15. May 2025\hfill Matthias Hefel


\end{document}
